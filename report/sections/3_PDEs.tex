Having assessed neural network performance across a range of ordinary differential equation problems,
we now turn to a more challenging class: partial differential equations (PDEs). In this section, 
we consider a single representative elliptic PDE posed on a two-dimensional domain:
\[
\begin{aligned}
    -\nabla^2 u(x, y) &= f(x, y), \quad \text{for } (x, y) \in \Omega := (0,1)^2, \\
    u(x, y) &= 0, \quad \text{for } (x, y) \in \partial \Omega,
\end{aligned}
\]
where the forcing function is given by
\[
f(x, y) = 13\pi^2 \sin(2\pi x)\sin(3\pi y).
\]
This problem admits an exact solution \( u(x, y) = \sin(2\pi x)\sin(3\pi y) \), which is smooth, bounded, 
and vanishes on the boundary of the domain.

Our goal is to assess the ability of neural networks to recover this solution from collocation data 
sampled over the interior and boundary of the domain. Based on prior results, we restrict our 
attention to the \(\tanh\) and Swish activation functions, as ReLU has been shown to perform poorly 
in earlier experiments. We follow a similar methodology to that used in the ODE setting: 
systematically varying depth and width of the network and evaluating performance using mean 
squared error over a dense evaluation grid.

In addition to architectural comparisons, we also investigate the robustness of the best-performing 
networks to data scarcity. Specifically, we assess how the approximation quality degrades as 
the number of training (collocation) points is reduced, thereby evaluating the sample efficiency 
of neural networks in recovering PDE solutions from limited information.

\subsection{Architectural Performance on the Poisson Problem}

